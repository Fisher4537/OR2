
\begin{abstract}
  This is the report of the entire Operational Research 2 course, hold by professor M. Fischetti in the Department of Engineering (DEI), University of Padua. The purpose is to introduce the theoretical concept regarding the Traveling Salesman Problem (TSP) as done during the course, to describe our group implemented methods and present some comparison between each other in term of execution time and solution cost.
  
  The theoretical part of the course is divided in two big block: in the first part are introduced the exact methods to resolve the TSP, in particular some optimization of the Simplex Algorithm and its implementation with a licensed software: CPLEX; in the second part are presented different heuristics which goal is to resolve larger instances of the TSP in acceptable time at the expenses of solution cost.
  
  Some Simplex Algorithm concepts related with TSP may be recalled, however it's not the purpose of this report to introduce Operational Research field, therefore Simplex concept as: incumbent, convex hull, continuous relaxation, branching variable, etc. are consider well known. To deepen in these basic concept it's suggested the theory book of the course \cite{fischetti2019introduction}.
  
  The most interesting part of this report is the application of Mixed Integer Programming methods (and the CPLEX software) to resolve a specific Linear Programming problem (TSP) and the comparison of exact methods results with heuristics in terms of execution time and solution cost. 
  All the code produced during the course is available in this public GitHub repository: \href{https://github.com/Fisher4537/OR2.git}{github.com/Fisher4537/OR2.git}.
  
  In the end of the report there is the appendix with the test set description (sec. \ref{sec:testset}), the method description used to create the performance profile plots (sec. \ref{sec:performance_meausure}) and the procedure to install CGAL library on windows.
\end{abstract}
