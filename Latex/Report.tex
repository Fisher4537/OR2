\documentclass[12pt]{article}
\usepackage[english]{babel}
\usepackage[utf8]{inputenc}
\usepackage[T1]{fontenc}
\usepackage{graphicx}
\usepackage{cite}
\usepackage{float}
\usepackage{amsfonts}
\usepackage{amsmath}
\usepackage{geometry}
\usepackage{braket}
\graphicspath{{IMAGES}}
\newtheorem{tab}{Tabla}
\renewcommand\tablename{Tabla}
\setlength{\columnsep}{7mm}
\renewcommand\tablename{Tabla}
\renewcommand\contentsname{\'Indice}
\setlength{\topmargin}{-0.75in} \setlength{\textheight}{9.25in}
\setlength{\oddsidemargin}{0.0in} \setlength{\evensidemargin}{0.0in}
\setlength{\textwidth}{6.5in}


\begin{document}

\title{Thesis of Operational Research 2}
\author{Luca Perali 1237770}


\twocolumn[
\begin{@twocolumnfalse}
\maketitle
\vspace*{-0.8cm}
\begin{center}\rule{0.9\textwidth}{0.1mm} \end{center}
\begin{abstract}
\normalsize This is the presentation of the problem and code managed in Operational Research 2, the Computer Engineering course of the University of Padua.
\begin{center}\rule{0.9\textwidth}{0.1mm} \end{center}
\vspace*{0.3cm}
\end{abstract}
\end{@twocolumnfalse}
]

\section{Travelling Salesman Problem (simmetric version)}
During the course the well-known Traveling Salesman Problem (TSP) was analyzed in its symmetrical version.
This is a proposed linear programming model which resolve the problem:
\[
\begin{cases}
min \sum_{ e\in E } c_ex_e \\
\sum_{e\in \delta (v) } x_e = 2, \forall v \in V \\
\sum_{e\in E(S) } x_e \le |S|-1, \forall S \subset V, |S| \ge 3 \\
x_e \in \Set{0,1}, \forall v \in V
\end{cases}
\]

\cite{}

\bibliographystyle{ieeetr}
\bibliography{Report}
\end{document}
