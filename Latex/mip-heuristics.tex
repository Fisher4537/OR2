\chapter{MIP Heuristics}
Mixed Integer Program Heuristics can be applied in each MIP problem independently from the context. These do not consider the specific formulation of the TSP problem however they could have good applicability. In this report two MIP Heuristics will be discussed and compared with the metaheuristics.

\section{Hard-fixing}
A simple idea to reduce the complexity of the problem is to get an initial solution like an incumbent, fix some active edges and resolve the subproblem with CPLEX. This is exactly the idea behind hard fixing approach.\\
The implementation proposed is applied in sTSP problem, with the optimization of the general callback (\texttt{subtour\_callback\_general}).
The algorithm can be divided in steps:
\begin{enumerate} \label{hard-fix-step}
	\item Calculate an initial solution: in our implementation this is done by CPLEX with \texttt{subtour\_callba \\ ck\_general} and \texttt{CPX\_PARAM\_INTSOLLIM} set to 1. When the first incumbent is available, the optimization terminate and a solution is obtained.
	\item Fix a percentage of edges (fixing rate $ f_r $): edges are fixed using \texttt{CPXchgbds} method that change the upper and lower bound of the decision variables. The fixing percentage is an important parameter: fixing too much edges leads to a fast resolution of the subproblem, however increase the risk of obtaining the same solution and fall in a loop. Fixing too low edges involves slower resolution. In the first iteration $ f_r = 0.9 $.
	\item CPLEX optimization with time limit: After the fixing phase, the problem is optimized with \texttt{CPX\_mipopt} and a short time limit is set (in our implementation $50 $s).
	\item Fixing rate update: after the last phase, if the returned solution is improved better than a fixed gap ($  good\_gap $), it is considered a good solution and the $ f_r $ increases by a constant ($ incr\_f_r $) until a max ($ max\_f_r $), otherwise if the new solution is not increased enough (less than $ optimal\_gap $), then $ f_r $ decreases of a constant ($ decr\_f_r $) until a min ($ min\_f_r $). Note that $ good\_gap $ and $ optimal\_gap $ are expressed as fraction of the current best lower bound.
	\item Check end condition: if the time limit is reached or $ f_r = 0.0 $ and the solution is not improved in the last iteration, then the solution is returned. In the second case, the best solution is found. If no ending condition is satisfied, the algorithm continue with point 3.
\end{enumerate}

\section{Local-Branching}
The Local Branching is a relatively recent algorithm developed by Matteo Fischetti and Andrea Lodi, who wanted to propose an alternative method to Hard Fixing. The Hard Fixing idea, as stated above, fix randomly a percentage of edges and optimize the problem, Local Branching, instead, let the MIP model to fix the percentage of edges automatically and than optimize. \\
Consider $ x^H $ as non optimal tour and $ \chi $ the fraction of edges to fix, Local Branching add
\begin{equation}
 \sum_{ e\in E: x_e^H = 1 } x_e \ge \chi n
\end{equation}
to automatically fix the desired number of edges. \\
In this paper, the structure of the algorithm is very similar to that of Hard Fixing. Initialization is analogous to Point 1 (\ref{hard-fix-step})
In the CPLEX optimization with time limit (Point 3), the time limit is a little higher and is set to 300 according to the choice of the fixing rate update. Thus, $ \chi n $ assume values in $ \{3, 5, 10, 15, 20\}$. It starts from the first value of 3 which is kept until better solutions are found. It is updated to the next value as soon as the algorithm returns a non-improving solution. This choice is motivated by the idea that keeping a low fixing rate allows you to find a solution in a short time and therefore not reach the time limit, which instead is set with a higher value to allow a longer search when the value increase. If the last value is reached and there is still no improvement solution, then the value is doubled until either a better solution is found or the number of nodes of the problem is reached. In the latter case, it is assigned the entire remaining time to find the best tour. In fig. \ref{fig:local_branching_a280} there is an example of the lower bound trend executed on \textit{a280.tsp} instance.\\
\begin{figure}[!h]
	\centering
	\includegraphics[width=0.5\columnwidth]{../res/local_branching_a280.png}
	\caption{Solution cost profile.}
	\label{fig:local_branching_a280}
\end{figure}

The exit conditions evaluate both the time limit and the gap calculated by CPLEX, obtainable through the \texttt{generic\_callback} using \texttt{CPX\_CALLBACKCONTEXT\_GLOBAL\_PROGRESS} as \texttt{Context}. 
For a detailed explanation of the reasons for this algorithm refer to \cite{article}.


\begin{figure}[h]
	\centering
	\begin{subfigure}{\columnwidth}
		\includegraphics[width=\columnwidth]{../res/Lmip_meta_LA_time.png}
		\caption{}
		\label{fig:Lmip_meta_LA_time}
	\end{subfigure}
	\begin{subfigure}{\columnwidth}
		\includegraphics[width=\columnwidth]{../res/Lmip_meta_LA_lb.png}
		\caption{}
		\label{fig:Lmip_meta_LA_lb}
	\end{subfigure}
\caption{Comparison of mip heuristics and meta heuristics. \texttt{n\_greedy\_best\_two\_opt} is the first part of \texttt{vns} therefore it is faster, however the last one introduce a little improvement in the solutions cost. Most of the times \texttt{hard\_fixing} can find the best solution of sTSP, indeed it find the shorter tours of the analyzed heuristics. }
\label{fig:Lmip_meta_LA}
\end{figure}

%\section{RINS}
%\section{Feasibility Pump}
%\section{Proximity Search}
%\section{Polishing}
