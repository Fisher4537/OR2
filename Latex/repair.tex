\chapter{Repair}
\section{Patching} \label{section:patching}
The Patching heuristic is a way to merge together multiple tours and obtain a single one.
The algorithm proposed is an iteration of a \texttt{single\_patch()} method which merge only one pair of tours. Note that an isolated node is the tour with minimum length therefore \texttt{patching()} can also resolve a TSP problem. \\
The proposed algorithm work through the 6 step presented in fig \ref{fig:single_patch}.
\begin{figure}[!h]
	\begin{subfigure}{.26\columnwidth}
		\includegraphics[width=\columnwidth]{img/patching1.png}
		\caption{The single patch input instance.}
		\label{fig:patching1}
	\end{subfigure}
\hfill%
	\begin{subfigure}{.26\columnwidth}
		\includegraphics[width=\columnwidth]{img/patching2.png}
		\caption{Node $ 3 $ is chosen randomly from a set of subtours.}
		\label{fig:patching2}
	\end{subfigure}
\hfill%
	\begin{subfigure}{.26\columnwidth}
		\includegraphics[width=\columnwidth]{img/patching3.png}
		\caption{It is selected the closer node to $ 3 $ which is a different tour (node $ 4 $).}
		\label{fig:patching3}
	\end{subfigure}
	\begin{subfigure}{.26\columnwidth}
		\includegraphics[width=\columnwidth]{img/patching4.png}
		\caption{Iterating throw the first tour nodes, it is selected the node which minimize the next merge cost.}
		\label{fig:patching4}
	\end{subfigure}
\hfill%
	\begin{subfigure}{.26\columnwidth}
		\includegraphics[width=\columnwidth]{img/patching5.png}
		\caption{Merge operation.}
		\label{fig:patching5}
	\end{subfigure}
\hfill%
	\begin{subfigure}{.26\columnwidth}
		\includegraphics[width=\columnwidth]{img/patching6.png}
		\caption{The final result.}
		\label{fig:patching6}
	\end{subfigure}
\caption{Single patch move algorithm}
\label{fig:single_patch}
\end{figure}\\
A \texttt{succ} structure is use to represent the input tours, which for each  $ i = 1, .., n = |V|$ is defined as: 
\begin{equation}
\texttt{succ[i] = } \begin{cases}
 \text{"\texttt{i} successor"} & \text{if \texttt{i} is in a tour,}\\
 \text{\texttt{-1}} & \text{otherwise.}
\end{cases}
\end{equation} \\
In the \texttt{succ} structure, each edge could be considered as oriented: if \texttt{succ[i] = j} than \texttt{i} $\rightarrow$ \texttt{j} is the considered edge orientation. \\
The merge phase is the core of the method. To avoid a large number of intersection between edges, which is a sign of non optimal solution, all the tour are initialize in clockwise orientation and the merge phase keep the property in the output tour. An example of merge is done in fig \ref{fig:patching_merge}.
\begin{figure}[!h]
	\begin{subfigure}{.26\columnwidth}
		\includegraphics[width=\columnwidth]{img/patching_merge1.png}
		\caption{}
		\label{fig:patching_merge1}
	\end{subfigure}
	\hfill%
	\begin{subfigure}{.26\columnwidth}
		\includegraphics[width=\columnwidth]{img/patching_merge2.png}
		\caption{}
		\label{fig:patching_merge2}
	\end{subfigure}
	\hfill%
	\begin{subfigure}{.26\columnwidth}
		\includegraphics[width=\columnwidth]{img/patching_merge3.png}
		\caption{}
		\label{fig:patching_merge3}
	\end{subfigure}

	\caption{Merge phase in detail.}
	\label{fig:patching_merge}
\end{figure}\\

\begin{table}[h]
	\centering
	\caption{The \texttt{succ} structure for the tour in figure \ref{fig:patching_merge}}
	\begin{tabular}{clcccccc}
		\multirow{2}{*}{\ref{fig:patching_merge1})} 	& \texttt{i:}		& 1 & 2 & 3 & 4 & 5 & 6 \\
														& \texttt{succ[i]:}	& 2 & 3 & 1 & 5 & 6 & 4 \\
														&		   			&   &   &   &   &   &   \\
		\multirow{2}{*}{\ref{fig:patching_merge3})} 	& \texttt{i:}		& 1 & 2 & 3 & 4 & 5 & 6 \\
														& \texttt{succ[i]:}	& 5 & 3 & 1 & 2 & 6 & 4 \\
	\end{tabular}
\end{table}
To keep the clockwise orientation of the tour there is a special consideration have to be done to merge an isolated node with a tour of two nodes. In the final tour the clockwise property is checked if the condition is satisfied:
$ (x_3 - x_1)*(y_2 - y_1) < (y_3 - y_1)*(x_2 - x_1) $,
otherwise the tour is reversed.


\begin{figure}[!h]
	\hfill%
	\begin{subfigure}{.2\columnwidth}
		\includegraphics[width=\columnwidth]{img/patching_merge_clockwise1.png}
		\caption{}
		\label{fig:patching_merge_clockwise1}
	\end{subfigure}
\hfill%
	\begin{subfigure}{.46\columnwidth}
		\includegraphics[width=\columnwidth]{img/patching_merge_clockwise2.png}
		\caption{}
		\label{fig:patching_merge_clockwise2}
	\end{subfigure}
\hfill%
	\caption{Special case of merge phase that create a tour with 3 edges.}
	\label{fig:patching_merge_clockwise}
\end{figure}
